\section{대한제국, 일본의 식민지로 전락하다}

\subsection{헤이그 밀사파견 사건과 정미조약 (1907)}

\subsubsection*{제2차 만국평화회의 개최 (1907)}

\begin{itemize}
    \item 제1차 만국평화회의는 '99년 개최, 국제분쟁 300여건 해결
    \item 제2차 회의는 '05년 개최 예정, 러일전쟁으로 '07년 6월 개최
    \begin{itemize}
        \item 겉으로는 세계 평화와 군비축소 논의가 목표
        \item 실제로는 열강들의 약소국 분할 지배를 위한 회의
        \item 1차 회의보다 참여국 수 증가, 중남미 국가 참여 증가 $\Rightarrow$ 미국의 영향력 국제적 확산
    \end{itemize}
\end{itemize}

\subsubsection*{헤이그 밀사 파견 (1907)}

\begin{itemize}
    \item 일본의 강압적인 보호조약 체결과 탄압을 열강의 지원을 받아 회복하려는 시도
    \item 한국이 러시아로부터 만국평화회의 초청장 받음 ('05. 10. 9) $\Rightarrow$ 이를 근거로 특사 파견
    \item 을사늑약 체결로 외교권 박탈 ('05. 11. 17)
    \item 고종, 헤이그특사 접견 ('07. 4. 20) $\Rightarrow$ 서울역 출발 (4. 22) $\Rightarrow$ 헤이그 도착 (6. 25)
    \item 열강은 한국의 대표권을 인정하지 않음 $\Rightarrow$ 만국평화 회의장 입장 실패
    \item 만국평화회의보와의 인터뷰 ('07. 7. 5): 을사늑약은 무력에 의해 체결된 불법적인 조약임을 주장
    \item 이준 열사의 사망
    \begin{itemize}
        \item 회의 참여가 거절되자 할복, 자살하였다고 보도됨 (이씨자살설)
        \item 국내에 큰 반향, 대일저항 증폭
    \end{itemize}
\end{itemize}

\subsubsection*{고종의 강제 퇴위 및 정미조약}

\begin{itemize}
    \item 통감 이토 히로부미, 헤이그 밀사사건이 을사늑약 위반임을 주장
    \item 이토, 이완용, 송병준 등의 책임 추궁 및 양위 주장
    \item 순종에게 황제 대리 윤허 $\Rightarrow$ 일본은 타국에게 황제 교체 통보 (7. 20)
    \item 정미조약 체결 (7. 24): 대한제국이 실질적인 식민지로 전락하는 조약
    \begin{itemize}
        \item 1조: 한국 정부은 시정개선에 관해 통감의 \textbf{지도}를 받을 것
        \item 2조: 한국 정부의 법령제정 및 중요한 행정상 처분은 미리 통감의 승인을 거침
        \item 3조: 한국의 사법사무(재판)는 보통의 행정 사무와 구분
        \begin{itemize}
            \item 재판 등을 일본이 수행
        \end{itemize}
        \item 4조: 한국 고위관리의 임면은 통감 동의 하에 집행
        \item 5조: 한국 정부는 통감이 추천하는 일본인을 한국 관리에 임명
        \item 6조: 한국 정부는 통감 동의 없이 외국인을 용빙하지 않음
    \end{itemize}
    \item 대한제국 군대 해산 (8. 1)
\end{itemize}

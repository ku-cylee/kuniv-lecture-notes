\section{1904년 대한제국과 러일전쟁, 그리고 동아시아}

\subsection{러일전쟁과 열강의 동아시아 재편}

\subsubsection*{러일전쟁 직전 동아시아 세계}

\begin{itemize}
    \item 중국
    \begin{itemize}
        \item 열강의 중국 분할 $\Rightarrow$ 중국 내 반외세 감정 폭발 $\Rightarrow$ 의화단운동 발생
        \item 열강의 의화단운동 진압 $\Rightarrow$ 신축각국조약 ('01. 9)
        \item 러시아: 만주지역에 군대 주둔, 영향력 강화
    \end{itemize}
    \item 영, 일: 제1차 영일동맹 ('02. 1)
    \begin{itemize}
        \item 전문: 청, 한국의 독립, 영토 보전 유지 및 두 나라에서 각국의 상공업이 균등한 기회 명시
        \item 1조: 영국은 청에서, 일본은 한국에서 취하고 있는 각별한 이익 보호
        \item 열강 중 가장 후발주자였던 일본이 가장 선발주자였던 영국과 동맹 체결
        \item 군비 확장, 국가 예산 3배 증가 $\Rightarrow$ 러시아 상대로 한국에 대한 우위를 점하기 위한 준비
    \end{itemize}
\end{itemize}

\subsubsection*{러일전쟁 (1904 - 05)}

\begin{itemize}
    \item 대한제국 대외중립선포 ('04. 1. 23)
    \item 여순항의 러시아 극동함대 기습 (2. 8), 인천항의 러시아 전함 공격 (2. 9)
    \item 일본의 선전포고 (2. 10)
    \item 일본의 강요로 한일의정서 체결 (2. 23)
    \begin{itemize}
        \item 1조: 한국은 일본을 확신하고 시정 개선에 대한 충고 수용
        \item 2조: 일본은 한국 황실을 안전강령(安全康寧)하게 한다
        \item 3조: 일본은 한국의 독립과 영토 보전을 보증
        \item 4조: 외세 침략, 내란 등으로 황실의 안녕, 영토 보전에 위험 $\Rightarrow$ 일본은 조치를 취할 수 있음
        \begin{itemize}
            \item 한국은 일본에 편의 제공 의무, 일본은 정황에 따라 전략상 필요한 지점을 차지 가능
            \item 러시아의 침략, 동학농민전쟁 등의 봉기 등을 상정
        \end{itemize}
        \item 한국은 러일전쟁 중 일본의 군사동맹국으로써 행동할 수밖에 없었음
    \end{itemize}
    \item 한국주차군 설립 (3. 10) $\Rightarrow$ 한국주차군군율 제정 (7. 2)
    \begin{itemize}
        \item 1조: 군용전선 $\cdot$ 철도 파손 시 사형
        \item 2조: 파손범을 알고도 은닉하는 경우 사형
        \item 5조: 군용전선 $\cdot$ 철도의 보호는 가설된 마을 전체의 인민 책임
        \item 6조: 군용전선 $\cdot$ 철도의 파손범이 체포되지 않은 경우 당일의 보호위원을 처벌
        \item 8조: 실수로 군용전선 $\cdot$ 철도를 파손한 자는 처벌, 침구와 음식은 본인 부담
        \item 민간인에게도 적용되는 군율
    \end{itemize}
    \item 일본 내각의 대한방침, 대한시설강령 결정 (5. 31) $\Rightarrow$ 전후 한국 지배를 위한 정책 마련
    \item 제1차 한일협약 체결 (8. 22)
    \begin{itemize}
        \item 일본이 추천하는 일본인 1인을 재정고문, 외국인 1인을 외교고문으로 초빙
    \end{itemize}
    \item 여순항, 봉천전투에서 일본 승리 ('05. 1 \textasciitilde{} 2), 쓰시마해전에서 일본 승리 ('05. 5)
    \item 열강의 동아시아 분할 조정
    \begin{itemize}
        \item 가쓰라 - 태프트 밀약 (미-일, 7. 27): 필리핀/한국에 대한 상호 양해
        \item 제2차 영일동망 (영-일, 8.12): 인도/한국에 대한 상호 양해
        \item 포츠머스 조약 (러-일, 9. 5): 한국에 대한 일본의 보호 승인
        \item 삼국간섭이 다시 일어나지 않게끔 사전에 교섭을 진행함
    \end{itemize}
\end{itemize}

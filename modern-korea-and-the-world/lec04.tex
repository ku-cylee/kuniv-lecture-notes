\section{열강, 조선의 운명을 논하다}

\subsection{갑신정변과 중 $\cdot$ 일의 대결}

\subsubsection*{갑신정변 (甲申政變)}
\begin{itemize}
    \item 1884년 12월 4일 발생
    \item 김옥균, 박영효, 서재필, 서광범, 홍영식 등 급진개화파가 시도한 무력 정변
\end{itemize}

\subsubsection*{배경}
\begin{itemize}
    \item 국내적 요인
    \begin{itemize}
        \item 1880년대 전후 조선 정부의 개혁, 개방 정책 $\rightarrow$ 전환국, 기기창, 우정국 등의 개화 기구를 창설
        \item 개화정책 방안의 분화: 급진개화파(일본 모델) vs 온건개화파(중국 모델)
    \end{itemize}
    \item 국제적 요인: 청불전쟁('84-'85)
    \begin{itemize}
        \item 베트남의 종주권을 둘러싼 청과 프랑스의 전쟁
        \item 청의 전세가 불리해지며 주조선 청군의 절반을 철수시킴.
        \item 일본과 프랑스가 동맹설 (불일동맹설)
        \item 이로 인해 '84.10 이후 일본의 대조선정책이 적극적으로 변화함
        \item 다케조에 신이치로 주조선 일본공사가 귀임, 정변지원에 대한 강한 암시
    \end{itemize}
\end{itemize}

\subsubsection*{쟁점: 자율론과 타율론}
\begin{itemize}
    \item 자율론: 개화파의 자주적인 개혁 추진
    \item 타율론: 일본 등 외세 의존적 성향
\end{itemize}

\subsubsection*{진행}
\begin{itemize}
    \item 12/4: 발생
    \begin{itemize}
        \item 우정국: 현대의 우체국과 같은 관청
        \begin{itemize}
            \item 현재 서울 종로구에 조계사 위치. 경복궁, 창경궁 등의 궁궐이 아주 가까운 곳에 있음
            \item 국왕을 확보하여 정치적 정당성을 가지기 용이하였음
        \end{itemize}
        \item 우정국 개국 축하파티에 참석한 반대파를 제거하고, 궁궐로 진출
        \item 우정국 축하파티에는 주요 국가 외교관과 조선군 지휘관들을 초빙
        \begin{itemize}
            \item 정변 직후 열강의 승인과 기존 군사력 제거를 위함
            \item 갑신일록에 따르면, 개화파는 이전부터 미국, 영국 등 열강과의 긴밀한 관계를 가지며 정변을 추진함
            \begin{itemize}
                \item 열강 외교관들은 갑신정변을 미룰 것을 제안
                \item 열강 공사로부터 정변을 지지받는 것이 매우 중요했음을 보여줌
            \end{itemize}
        \end{itemize}
    \end{itemize}
    \item 12/5: 신정부의 각료 명단 발표, 개화파와 종친세력의 연립정부 성격
    \begin{itemize}
        \item 외교: 조공허례 폐지, 중국에 대한 자주(自主) 추구 (1조)
        \begin{itemize}
            \item 대원군을 가까운 시일 내에 모셔올 것을 제시함
        \end{itemize}
        \item 정치: 인재 등용(4조) ,정치기구 통합 및 효율화(13, 14조)
        \item 경제: 지조법 개혁(3조), 재정일원화(12조), 환곡 개혁(6조)
        \begin{itemize}
            \item 3조: 지조(납세)법 개혁 $\rightarrow$ 백성 구제
        \end{itemize}
        \item 군사: 군사제도 정비(11조)
        \item 사회: 인민평등권 보장(2조), 치안 강화(8조)
        \begin{itemize}
            \item 2조: 문벌 폐지, 인민평등 권리 제정, 능력에 따라 관(官)을 택함
        \end{itemize}
    \end{itemize}
    \item 12.6: \textbf{3일천하}
    \begin{itemize}
        \item 청군 1,500명 창덕궁 진입, 일본군 200여명과 개화파군 50여명과 전투
        \item 개화파는 패배, 다수 사망, 김옥균 및 박영효 등 일부는 일본 망명.
    \end{itemize}
\end{itemize}

\subsubsection*{한계}
\begin{itemize}
    \item 개혁 주체: 정변을 주도한 인물들이 너무 어렸음 (평균 25.4세)
    \item 추진 방법: 국왕의 앞에서 측근들을 제거해버리는 형태의 급진적인 정권 탈취가 수용되기 어려웠음
    \item 외세와의 관계: 일본의 지원을 받았다는 인식으로 인해 지지받지 못함
\end{itemize}

\subsubsection*{국제 정세에서의 의미}
\begin{itemize}
    \item 갑신정변은 국내뿐만 아닌 국제 관계에서도 중대한 사건이었음.
    \begin{itemize}
        \item 열강 세력들의 다툼과 조정이 일어나던 시기
        \item 조선의 정변이 곧바로 청일간의 대결로 비화되는 계기가 됨
    \end{itemize}
\end{itemize}

\subsection{조선을 둘러싼 열강의 동아시아 정책}

\subsubsection*{갑신정변 이후 중, 일과의 조약}
\begin{itemize}
    \item 한성조약('85. 1. 9): 조-일
    \item 천진조약('85. 4. 18): 청-일
    \begin{itemize}
        \item 양국은 4개월 내에 조선에서 군대 철수
        \item 양국은 조선에 군사고문을 파견하지 않음
        \item 조선에 출병이 필요한 경우 상호간 통지 후 파병
        \item 조선에서의 양군 충돌 방지를 위함
    \end{itemize}
\end{itemize}

\subsubsection*{거문도 점령 사건}
\begin{itemize}
    \item 영국이 남해의 거문도를 사전 통고 없이 점령한 사건 ('85. 4. 15 - '87. 2. 27)
    \item 청일간 천진조약 협상 기간 중 발생하였으나, 갑신정변과는 무관
    \item 세계적 단위의 영-러간 대결 구도가 조선까지 영향을 미친 사건
    \begin{itemize}
        \item 1차 아프간 전쟁 ('39 - '42): 아프가니스탄
        \item 크림 전쟁 ('53 - '56): 터키
        \item 이리 위기 ('71 - '81): 카자흐스탄
        \item 펜제 사건 ('85): 아프가니스탄
        \item 러시아의 남하 견제를 위해 거문도 선제 점령
    \end{itemize}
    \item 거문도의 전략적 가치
    \begin{itemize}
        \item '50년대부터 영, 프가 일찍이 주목하였음
        \item 블라디보스토크, 나가사키, 상하이 등을 동시에 견제할 수 있는 요충지
        \item 영국은 거문도를 장기적 군사기지화 계획
    \end{itemize}
    \item 강화도 조약 10년 이후, 조선이 열강들의 세계 분쟁 중심지로 부상
    \item 조선의 의사와 무관하게 양국간 세력분쟁의 최전선이 됨
    \item 조선의 위상은 국내정치 뿐만 아니라 열강의 세계정책과 연계됨
\end{itemize}

\subsubsection*{한반도 문제의 해결 방안}
\begin{itemize}
    \item 한반도 중립론
    \begin{itemize}
        \item 주조선 독일 부영사 부들러(Budeler)가 제시
        \item 유길준이 중립론(中立論) 주장
        \item 중립론은 청에 의존하여 처리해야 함을 주장
    \end{itemize}
    \item 탈아론 (탈아시아론)
    \begin{itemize}
        \item 후쿠자와 유키치가 주장
        \item '80초 조선 개화파에 대한 가장 직접적인 후원자
        \item 일본은 그간 같은 동아시아 국가로써 조선을 바라봄
        \item 이제는 일본 자신을 서양의 국가로 생각하고 서양 열강이 조선을 대하는 방식대로 대해야 함
        \item 현재까지도 일본의 대외정책을 논의하는데 중요한 인식
        \item 일본 재야의 대표적인 아시아 인식을 보여주는 자료
    \end{itemize}
    \item 외교정략론
    \begin{itemize}
        \item 일본 의회 개설 당시 수상 야마가타 아리토모가 주장
        \item 국가 독립과 자위의 방법: 주권선(영토)을 지키고, 이익선(주권선의 안위와 긴밀하게 관계되는 구역)을 방어
        \item 주권선을 온전히 지키기 위해서는 주권선과 가장 가까운 이익선이 자국의 세력에 들어와야 함
        \item 일본 정부의 아시아 인식을 보여주는 자료
        \item 청일전쟁, 러일전쟁이 일어나는 계기 $\rightarrow$ 피해는 조선의 몫
    \end{itemize}
\end{itemize}

\section{대한제국, 일본의 식민지로 전락하다}

\subsection{헤이그 밀사파견 사건과 정미조약 (1907)}

\subsubsection*{제2차 만국평화회의 개최 (1907)}

\begin{itemize}
    \item 제1차 만국평화회의는 '99년 개최, 국제분쟁 300여건 해결
    \item 제2차 회의는 '05년 개최 예정, 러일전쟁으로 '07년 6월 개최
    \begin{itemize}
        \item 겉으로는 세계 평화와 군비축소 논의가 목표
        \item 실제로는 열강들의 약소국 분할 지배를 위한 회의
        \item 1차 회의보다 참여국 수 증가, 중남미 국가 참여 증가 $\Rightarrow$ 미국의 국제적 영향력 확산
    \end{itemize}
\end{itemize}

\subsubsection*{헤이그 밀사 파견 (1907)}

\begin{itemize}
    \item 일본의 강압적인 보호조약 체결과 탄압을 열강의 지원을 받아 회복하려는 시도
    \item 한국이 러시아로부터 만국평화회의 초청장 받음 ('05. 10. 9) $\Rightarrow$ 이를 근거로 특사 파견
    \item 을사늑약 체결로 외교권 박탈 ('05. 11. 17)
    \item 고종, 헤이그특사 접견 ('07. 4. 20) $\Rightarrow$ 서울역 출발 (4. 22) $\Rightarrow$ 헤이그 도착 (6. 25)
    \item 열강은 한국의 대표권을 인정하지 않음 $\Rightarrow$ 만국평화 회의장 입장 실패
    \item 만국평화회의보와의 인터뷰 ('07. 7. 5): 을사늑약은 무력에 의해 체결된 불법적인 조약임을 주장
    \item 이준 열사의 사망
    \begin{itemize}
        \item 회의 참여가 거절되자 할복, 자살하였다고 보도됨 (이씨자살설)
        \item 국내에 큰 반향, 대일저항 증폭
    \end{itemize}
\end{itemize}

\subsubsection*{고종의 강제 퇴위 및 정미조약}

\begin{itemize}
    \item 통감 이토 히로부미, 헤이그 밀사사건이 을사늑약 위반임을 주장
    \item 이토, 이완용, 송병준 등의 책임 추궁 및 양위 주장
    \item 순종에게 황제 대리 윤허 $\Rightarrow$ 일본은 타국에게 황제 교체 통보 (7. 20)
    \item 정미조약 체결 (7. 24): 대한제국이 실질적인 식민지로 전락하는 조약
    \begin{itemize}
        \item 1조: 한국 정부은 시정개선에 관해 통감의 \textbf{지도}를 받을 것
        \item 2조: 한국 정부의 법령제정 및 중요한 행정상 처분은 미리 통감의 승인을 거침
        \item 3조: 한국의 사법사무(재판)는 보통의 행정 사무와 구분
        \begin{itemize}
            \item 재판 등을 일본이 수행
        \end{itemize}
        \item 4조: 한국 고위관리의 임면은 통감 동의 하에 집행
        \item 5조: 한국 정부는 통감이 추천하는 일본인을 한국 관리에 임명
        \item 6조: 한국 정부는 통감 동의 없이 외국인을 용빙하지 않음
    \end{itemize}
    \item 대한제국 군대 해산 (8. 1)
\end{itemize}

\subsection{`병합'과 열강의 세계 분할}

\subsubsection*{한일합방을 위한 일본의 사전 조치}

\begin{itemize}
    \item 삼국간섭 이후 조선 지배에 대한 열강의 승인을 받기 위해 노력
    \item 일본은 열강에게 조선의 \textbf{자주 독립}을 공언해왔음 $\Rightarrow$ 열강의 동의 및 묵인 필요
    \begin{itemize}
        \item 시모노세키조약('95), 한일의정서('04), 포츠머스조약('05)
    \end{itemize}
    \item 미, 영, 프, 러의 동의 하에 병합조약 체결 (가쓰라-태프트 밀약 등)
\end{itemize}

\subsubsection*{안중근의 이토 히로부미 암살 사건}

\begin{itemize}
    \item 전 통감 이토 히로부미, 러시아 재무상 코코프초프가 만주에서의 이익 조정을 위한 경제 회담
    \item 하얼빈 역에서 이토 히로부미 사살 ('09. 10. 26)
    \item 하얼빈(러시아 조차지)에서 러시아 헌병대, 청 경찰에 체포 $\Rightarrow$ 뤼순(일본 조차지)로 이송
    \begin{itemize}
        \item 러시아 법정에 서는 것을 방지하기 위함
    \end{itemize}
    \item 사형 선고 ('10. 2. 14) $\Rightarrow$ 사형 집행, 안중근 의사 순국 ('10. 3. 26)
    \item \textless{}동양평화론\textgreater{} 집필: 안중근 의사의 동양 평화 추구 이상 및 사상이 잘 드러남
    \begin{itemize}
        \item 러일전쟁을 동양평화를 위한 전쟁, 황인종 · 백인종 간 전쟁으로 인식
        \item 삼국 공영론: 한, 중, 일 3국이 같이 외세에 대적해야 한다는 이론. 황성신문 등에 나타남
        \item 러일전쟁에서 일본의 승리 $\Rightarrow$ 한국 발전에 기여할 것으로 판단
        \item 이토가 '동양평화', '한국 독립'을 저버렸기 때문에 사살했다고 주장
        \item 한중일 3국의 연합 군대 편성, 상호 국가간 언어 습득, 3국 공동화폐 발행 제시
    \end{itemize}
\end{itemize}

\subsubsection*{병합조약 ('10. 8. 22)}

\begin{itemize}
    \item 한일 양국 황제는 상호 행복 증진, 동양 평화 영구 확보를 위해 병합 결정
    \item 1조: 한국 황제는 한국 전부에 관한 일체의 통치권을 완전히, 영구히 일본 황제에게 양여
    \item 2조: 일본 황제는 1조의 양여 수락, 병합을 승낙
    \item 3조: 한국 황제의 직계 가족에 대한 존칭, 명예 등을 보장하고 이를 위한 세비 공급 약속
    \item 4조: 3조 이외의 한국 황실에 대해서도 자금 공여 약속
    \item 5조: 병합에 훈공이 있다고 인정되는 자에게 작위 및 은금 수여
    \item 6조: 일본은 한국의 시정을 담임, 법규를 준수하는 자에 대한 충분한 보호 보장
    \item 7조: 일본은 신(新)제도를 존중하는 한국인을 관리로 등용
    \item 8조: 재가를 받은 날로부터 시행함
\end{itemize}

\subsubsection*{에필로그}

\begin{itemize}
    \item 전 세계는 식민 지배 국가(제국) / 식민 피지배 국가(식민지)로 분할
    \item 제1차 세계대전('14 - '18): 열강들의 세력 재분할 시도로 발생. 사상 30M명, 사망 9M명
    \item 파리 강화 회의 ('19) $\Rightarrow$ 베르사유 체제, 국제 연맹
    \item 제2차 세계대전('39 - '45): 열강들의 이익 분할 시도가 다시 충돌. 사망자  47M명
    \item 1800년: 아메리카 대륙 이외 대부분 지역은 독립국가
    \item 1914년: 중국, 중남미 등을 제외한 대부분 지역이 식민지화
    \item 오늘날 동아시아 3국
    \begin{itemize}
        \item 3국의 GDP 합이 전 세계 20\%를 넘음
        \item 3국은 19C말 - 20C초 각기 처했던 역사적 조건이 달랐음
            $\Rightarrow$ 평화로운 교류, 발전 저해
        \item 3국 간 영토분쟁은 근대 시기 역사적 유산이 청산되지 못한 결과
    \end{itemize}
\end{itemize}
